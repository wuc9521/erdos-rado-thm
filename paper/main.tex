\documentclass{amsart}
\usepackage[T1]{fontenc}
\usepackage{amsthm}

\newcommand{\Erdos}{ERD\H{O}S}
\newcommand{\erdos}{Erd\H{o}s}
\theoremstyle{plain}
\newtheorem{theorem}{Theorem}[section]
\newtheorem{lemma}[theorem]{Lemma}
\newtheorem{corollary}[theorem]{Corollary}

\theoremstyle{definition}
\newtheorem{definition}[theorem]{Definition}
\newtheorem*{example}{Example}
\newtheorem{proposition}[theorem]{Proposition}
\newtheorem{notation}{Notation}


\title{A proof of \Erdos-Rado Theorem}
\author{Chentian Wu}
\date{}

\begin{document}
\maketitle

\begin{abstract}
We discuss some of the key ideas of Perelman's proof of  Poincaré's conjecture via the Hamilton program of using  the Ricci flow, from the perspective of the modern theory of nonlinear partial differential equations.
\end{abstract}

\tableofcontents
\newpage
\newpage
\section{Introduction}
\newpage
\section{Preliminary Set Theory}
The notion of regular uncountable cardinal is a generalization of the pigeonhole principle for uncountable sets.

\begin{definition}[ordinal]
    An \emph{ordinal} is a set $\alpha$ such that $\bigcup\alpha\subset\alpha$ and well-ordered by $\in$.
\end{definition}

\begin{definition}[cardinal]
    Assuming axiom of choice, a set $A$ is said to have cardinality $\kappa$ if there exists a bijection between $A$ and $\kappa$. $\aleph_0$ is the cardinality of the set of natural numbers. And it could be proved that $\aleph_0$ is the least infinite cardinal.
\end{definition}


\begin{definition}[regular cardinal]
    An uncountable cardinal number $\lambda$ is said to be \textit{regular} if for every $\kappa<\lambda$, every set $S$ of cardinal $\lambda$, and every function $f:S\to\kappa$, there exists $H\subset S$ of cardinal $\lambda$ such that the function $f$ is constant on $H$.
\end{definition}


\begin{definition}[successor cardinal]
    Let $\mu$ be a cardinal, define the \emph{successor cardinal} of $\mu$ to be the least cardinal strictly greater than $\mu$, denoted by $\mu^+$.
\end{definition}

\begin{proposition}
    Suppose $\lambda$ is an uncountable regular cardinal. For every family of sets $\{A_i\mid i<\lambda\}$ such that $i<j\iff A_i\subset A_j$, we have that for every $A\subset \bigcup_{}$
\end{proposition}
\begin{proposition}
    For every infinite cardinal $\mu$, the cardinal $\mu^+$ is regular.
\end{proposition}
\begin{proof}
    
\end{proof}


\begin{notation}
    Let $n$ be a positive integer, and let $S$ be a linearly ordered set. Denote the set $\{(j_1, j_2, \ldots, j_n)\mid j_1<j_2<\ldots<j_n\}$ by $[S]^n$. For a subset $H\subset S$, let
    \[
        [H]^n=\{(j_1, j_2, \ldots, j_n)\in H^n\mid j_1<j_2<\ldots<j_n\}.
    \]
\end{notation}

\begin{example}
    We can view $[\omega]^2$ as the complete undirected graph on $\omega$, and any pair of distinct elements in $\omega$ is an edge in the graph.
\end{example}


\begin{notation}
    $[S]^{<\omega}= \bigcup_{n<\omega}[S]^n$.
\end{notation}


\begin{definition}[Beth numbers]
    Beth numbers are defined by transfinite recursion as follows:
    \begin{enumerate}
        \item $\beth_0=\aleph_0$.
        \item $\beth_{\alpha+1}= 2^{\beth_\alpha}$.
        \item $\beth_\lambda=\displaystyle{\sup_{\alpha<\lambda}\{\beth_\alpha}\}$.
    \end{enumerate}
\end{definition}
\section{Ramsey's Theorem}


\begin{theorem}[Ramsey's Theorem]
Let \dots
\end{theorem}

\newpage
\section{\erdos-Rado Theorem}
\newpage
\section{References}
\end{document}
