\documentclass{amsart}
\usepackage[T1]{fontenc}
\usepackage{amsthm}
\usepackage{amssymb}

\newcommand{\Erdos}{ERD\H{O}S}
\newcommand{\erdos}{Erd\H{o}s}
\newcommand{\R}{\mathbb R}
\newcommand{\Q}{\mathbb Q}
\newcommand{\proves}{\vdash}
\newcommand{\tp}{\textsf{tp}}
\newcommand{\fml}{\textsf{fml}}
\newcommand{\bd}[1]{\boldsymbol{#1}}

\theoremstyle{plain}
\newtheorem{theorem}{Theorem}[section]
\newtheorem{lemma}[theorem]{Lemma}
\newtheorem{corollary}[theorem]{Corollary}

\theoremstyle{definition}
\newtheorem{definition}[theorem]{Definition}
\newtheorem*{example}{Example}
\newtheorem{proposition}[theorem]{Proposition}
\newtheorem{notation}{Notation}


\title{A proof of \Erdos-Rado Theorem}
\author{Chentian Wu}
\date{}

\begin{document}
\maketitle

\begin{abstract}
We discuss some of the key ideas of Perelman's proof of  Poincaré's conjecture via the Hamilton program of using  the Ricci flow, from the perspective of the modern theory of nonlinear partial differential equations.
\end{abstract}

\tableofcontents
\newpage
\input{sections/intro.tex}
\newpage
\section{Preliminary Set Theory}
The notion of regular uncountable cardinal is a generalization of the pigeonhole principle for uncountable sets.

\begin{definition}[ordinal]
    An \emph{ordinal} is a set $\alpha$ such that $\bigcup\alpha\subset\alpha$ and well-ordered by $\in$.
\end{definition}

\begin{definition}[cardinal]
    Assuming axiom of choice, a set $A$ is said to have cardinality $\kappa$ if there exists a bijection between $A$ and $\kappa$. $\aleph_0$ is the cardinality of the set of natural numbers. And it could be proved that $\aleph_0$ is the least infinite cardinal.
\end{definition}


\begin{definition}[regular cardinal]
    An uncountable cardinal number $\lambda$ is said to be \textit{regular} if for every $\kappa<\lambda$, every set $S$ of cardinal $\lambda$, and every function $f:S\to\kappa$, there exists $H\subset S$ of cardinal $\lambda$ such that the function $f$ is constant on $H$.
\end{definition}


\begin{definition}[successor cardinal]
    Let $\mu$ be a cardinal, define the \emph{successor cardinal} of $\mu$ to be the least cardinal strictly greater than $\mu$, denoted by $\mu^+$.
\end{definition}

\begin{proposition}
    Suppose $\lambda$ is an uncountable regular cardinal. For every family of sets $\{A_i\mid i<\lambda\}$ such that $i<j\iff A_i\subset A_j$, we have that for every $A\subset \bigcup_{}$
\end{proposition}
\begin{proposition}
    For every infinite cardinal $\mu$, the cardinal $\mu^+$ is regular.
\end{proposition}
\begin{proof}
    
\end{proof}


\begin{notation}
    Let $n$ be a positive integer, and let $S$ be a linearly ordered set. Denote the set $\{(j_1, j_2, \ldots, j_n)\mid j_1<j_2<\ldots<j_n\}$ by $[S]^n$. For a subset $H\subset S$, let
    \[
        [H]^n=\{(j_1, j_2, \ldots, j_n)\in H^n\mid j_1<j_2<\ldots<j_n\}.
    \]
\end{notation}

\begin{example}
    We can view $[\omega]^2$ as the complete undirected graph on $\omega$, and any pair of distinct elements in $\omega$ is an edge in the graph.
\end{example}


\begin{notation}
    $[S]^{<\omega}= \bigcup_{n<\omega}[S]^n$.
\end{notation}


\begin{definition}[Beth numbers]
    Beth numbers are defined by transfinite recursion as follows:
    \begin{enumerate}
        \item $\beth_0=\aleph_0$.
        \item $\beth_{\alpha+1}= 2^{\beth_\alpha}$.
        \item $\beth_\lambda=\displaystyle{\sup_{\alpha<\lambda}\{\beth_\alpha}\}$.
    \end{enumerate}
\end{definition}
\newpage
\section{Ramsey's Theorem}

\begin{notation}
    Let $n<\omega$ and suppose $\lambda, \kappa$ and $\mu$ are cardinal numbers (not necessarily infinite). We denote by $\lambda\to(\mu)_\kappa^n$ the following statement: \\
    For every set $S$ of cardinal $\lambda$ and every function $c:[S]^n\to\kappa$, there exists $H\subset S$ of cardinal $\mu$ and there exists $i_0<\kappa$ such that for every $\boldsymbol j\in[H]^n$, $c(\boldsymbol j)=i_0$. \\
    We denote by $\lambda\not\to(\mu)_\kappa^n$ if the above statement is false.
\end{notation}

In the example above, we call the function $c$ \emph{coloring with $\kappa$ colors}, and the set $H$ \emph{monochromatic set}. It's obvious that if we can understand the case of $n=2$, the other cases can be easily deduced.


\begin{theorem}[infinite Ramsey]
    $\aleph_0\to(\aleph_0)_2^2$
\end{theorem}

\begin{theorem}[Finite Ramsey Theorem]
    For every positive integers $m, n, k$, there exists a positive integer $l(m,n,k)$ such that $l(m,n,k)\to (m)_k^n$
\end{theorem}

\begin{lemma}
    Let $n,\lambda,\kappa$ and $\mu$ be cadinals. Suppose that $\lambda\to(\mu)_\kappa^n$ holds, then the following statements hold:
    \begin{enumerate}
        \item\label{lemma:lambda} If $\lambda\leq\lambda'$, then $\lambda'\to(\mu)_\kappa^n$ holds.
        \item If $\mu\geq\mu'$, then $\lambda\to(\mu')_\kappa^n$ holds.
        \item If $\kappa\geq\kappa'$, then $\lambda\to(\mu)_{\kappa'}^n$ holds.
        \item If $n\geq n'$, then $\lambda\to(\mu)_\kappa^{n'}$ holds.
    \end{enumerate}
\end{lemma}

\begin{theorem}[Sierpinski 1933]
    $\aleph_1$ is not weakly compact. That is, $\aleph_1\not\to(\aleph_1)_2^2$.
\end{theorem}
\begin{proof}
    By Cantor's theorem ($2^{\aleph_0}\geq \aleph_1$) and \ref{lemma:lambda}, it's enough to show that $2^{\aleph_0}\not\to(\aleph_1)_2^2$. Fix $<^*$ to be a well-ordering of the reals. Denote by $<_\R$ the usual ordering of the reals. For every $a<^*b\in\R$ define a coloring as follows
    \[
    F(a, b) := \begin{cases}
        0 & \text{if } a<_\R b \\
        1 & \text{o.w.}
    \end{cases}
    \]
    For the sake of contradiction, suppose there exists a set $S\subset\R$ of cardinal $\aleph_1$ such that $F$ is constant on $[S]^2$. 
    In case the constant value of $F$ is 0 then by definition of $F$ we have that $a<_\R b\iff a<^* b$. Since $S$ has cardinality $\aleph_1$ we can pick an $<^*$-increasing sequence $\{a_\alpha\mid \alpha<\aleph_1\}$ of elements of $S$. Now for every $\alpha<\omega$ define $I_\alpha=(a_\alpha, a_{\alpha+1})$. Since the order $<^*$ coincide on $S$ with the usual ordering of the reals and $\{a_\alpha\}$ is increasing, we have that $i<j<\aleph_1$ implies $I_i\cap I_j=\emptyset$. Since the rationals $\Q$ are dense in $\R$, for every $\alpha<\aleph_1$ we can pick a rational $q_\alpha\in I_\alpha$. Since $\Q$ is countable, there exists a rational $q$ such that $q=q_\alpha$ for uncountably many $\alpha$. But this contradicts the fact that $\{I_\alpha\}$ is a pairwise disjoint family. The case where the constant value of $F$ is 1 is similar.
\end{proof}

The above theorem shows that $\aleph_1\not\to(\aleph_1)_2^2$, so a natural question would be: Does there exists a cardinal $\lambda$ such that $\lambda\to(\aleph_1)_2^2$ holds? What we know is, if such a cardinal exists, it should be strictly greater than $\aleph_1$. 

\begin{definition}[weakly compactness]
    An uncountable $\chi$ is called \emph{weakly compact} if $\chi\to(\chi)_2^2$ holds.
\end{definition}

A natural question to ask is: Do weakly compact cardinals exist? The answer is yes, and the existence of weakly compact cardinals is independent of ZFC.

\begin{theorem}[ZFC]
    Let $\chi$ be an uncoutable cardinal. If $\chi$ is weakly compact, then the first order theory ZFC has a model (namely ZFC is consistent).
\end{theorem}

\begin{theorem}[Godel's Incompleteness Theorem]
    If ZFC is consistent, then ZFC is incomplete. That is, from ZFC alone is not enough to prove the consistency of ZFC.
\end{theorem}

From the above two facts, we get the following corollary immediately.
\begin{corollary}
    ZFC $\not\proves$ the existence of weakly compact cardinals.
\end{corollary}

\newpage
\section{\erdos-Rado Theorem}

\begin{definition}
    Let $\alpha$ be an ordinal and let $\lambda$ be a carinal. By induction on ordinals, we define the cardinal number $\beth_{\alpha}(\lambda)$ as follows:
    \begin{enumerate}
        \item $\beth_0(\lambda)=\lambda$ 
        \item if $\alpha$ is a limit ordinal, $\beth_\alpha(\lambda):=\displaystyle{\bigcup_{i<\alpha}\beth_i(\lambda)}$
        \item if $\alpha=\beta+1$, $\beth_\alpha(\lambda)=2^{\beth_\beta(\lambda)}$
    \end{enumerate}
    When $\lambda=\aleph_0$ we write $\beth_\alpha$ instead of $\beth_\alpha(\aleph_0)$.
\end{definition}

With the notation above, $\beth_1(\lambda)$ is just another way to write $2^\lambda$. An alternative way to state the generalized continuum hypothesis is for every ordinal $\alpha$, $\beth_\alpha=\aleph_\alpha$ holds.


\begin{theorem}[The \erdos-Rado Theorem]
    For every natural number $n$ and every infinite cardinal $\lambda$, we have that
    \[
    \beth_n(\lambda)^+\to(\lambda^+)_\lambda^{n+1}
    \]
\end{theorem}

Before staring the proof, we'll need to introduce a central model-theoretic concept which is the generalization of a first-order formula. 

\begin{definition}
    Let $L$ be a similarity type, let $M$ be an $L$-structure, $A\subset |M|$ and $\boldsymbol{a}\in|M|$. Then we define
    \[
    \tp(\boldsymbol{a}/A, M):=\{\varphi(\boldsymbol{x}; \boldsymbol{b})\mid \varphi(\boldsymbol{x};\boldsymbol{y})\in \fml(L),\boldsymbol{b}\in A,M\vDash\varphi[\boldsymbol{a};\boldsymbol{b}]\}
    \]
    We call $\tp(\boldsymbol{a}/A, M)$ the \emph{type} of $\boldsymbol{a}$ over $A$ in $M$.
\end{definition}


\begin{example}
    Suppose $L$ is the language of fields, abd suppose that $M$ is a field. Given $A\subset |M|$ and $\boldsymbol{a}\in |M|$, $\tp(\boldsymbol{a}/A,M)$ contains all the equations with the coefficients in $A$ in unknown $\boldsymbol{x}$ that $\boldsymbol{a}$ is a solution for.
\end{example}

\begin{definition}
    Let $L$ be a similarity type, let $M$ be an $L$-structure, $A\subset |M|$ and let $\b{x}$
\end{definition}
\input{sections/refs.tex}
\end{document}
